\documentclass[a4paper,11pt]{article}
\usepackage[utf8]{inputenc}
\usepackage[T2A]{fontenc}
\usepackage[russian,english]{babel}
\usepackage{geometry}
\usepackage{hyperref}
\usepackage{array}
\usepackage{enumitem}
\geometry{margin=1in}
\hypersetup{
    colorlinks=true,
    urlcolor=blue,
}

\begin{document}

\begin{center}
    {\LARGE \textbf{Куценко Егор}}\\
    \vspace{1mm}
    {\normalsize Backend-разработчик}\\
    \vspace{1mm}
    {\small \today}
\end{center}

\vspace{2mm}

\section*{Контактная информация}
\begin{itemize}[leftmargin=*, label={}]
    \item \textbf{Email:} \href{shotsoks@gmail.com}{shotsoks@gmail.com}
    \item \textbf{Телефон:} +7\,989\,514-05-02
    \item \textbf{TG:} @Shotso97
    \item \textbf{GitHub:} \href{https://github.com/Shotso2839}{https://github.com/Shotso2839}
\end{itemize}

\section*{О себе}
Опытный backend-разработчик с опытом разработки и проектирования высоконагруженных систем. Занимаюсь написанием высоконагруженных микросервисов на C++, Go Python, оптимизацией и базами данных, контейнеризации и настройкой CI/CD. Стремлюсь создавать надёжные, масштабируемые и поддерживаемые решения.

\section*{Образование}
\begin{itemize}[leftmargin=*, label={}]
    \item \textbf{НИУ ВШЭ}, Факультет компютерных наук КНАД\\
    Бакалавриат Сентябрь 2024 – Июнь 2028
\end{itemize}

\section*{Ключевые навыки}
Python (Django, FastAPI), Go, C++; REST, gRPC, Echo, Gin; PostgreSQL, MySQL, MongoDB, Redis, Kafka, RabbitMQ; Docker, Kubernetes, AWS (EC2, S3, RDS); Git, GitLab CI/CD, GitHub Actions, Prometheus, Grafana.


\section*{Опыт работы 5 лет}

\subsection*{Опыт при работе с собственными проектами}
\begin{itemize}[leftmargin=*]
    \item Проектирование и разработка микросервисов на Go (Gin).
    \item Реализация REST и gRPC API для веб- и мобильных клиентов.
    \item Настройка PostgreSQL (шардинг, репликация), оптимизация сложных SQL-запросов.
    \item Внедрение кэширования с Redis для ускорения ответов.
    \item Контейнеризация сервисов (Docker), оркестрация в Kubernetes на AWS.
    \item Настройка CI/CD в GitLab CI: тестирование, сборка, деплой.
    \item Мониторинг (Prometheus \& Grafana) и логирование (ELK-стек).
\end{itemize}

\end{document}